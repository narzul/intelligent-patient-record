\section{Design analysis and concept diagrams}
Description of issues related to the design of the product:
- Description of concepts, requirements and features of the product
- Review of motivations for making the design decisions
- Indicate the primary features of the design that are the most creative and original

\subsection{Materials}
For this project we got a bunch of different plast materials from RIAS, in order to find some material that might be cheaper, and better that acrylic, since acrylic have tendency to be brittle, this becomes worse when it has been laser cut.
all of the plastic was thermoplastic, which is a term that is used in the laser industri to indicate that it can be laser c ut.
for the laser cutter we made some simple figures to see what the result would be.
\subsubsection{PEHD}
The first material that we tryed to cut was, PEHD the cutting went fine, but there was some resedu left over from the cutting.
There ware some problems getting the resedu off the plastic after the cutting, also the material have a tendensy to hold it's form when bent.
\begin{figure}[h]
	\begin{center}
		\includegraphics[scale=0.1]{images/DSC_0048.JPG}
		\caption{\small {\it {This is a description}}} \label{fig:explode}
	\end{center}
\end{figure}
\subsubsection{RIALEN PP}
the cuttig had some big problems, one of them was that after 3 cutting rounds, the laser still haven't cut through the material, which ment we had to let it be, and not trying to cut in that material. 
\begin{figure}[h]
	\begin{center}
		\includegraphics[scale=0.1]{images/DSC_0053.JPG}
		\caption{\small {\it {This is a description}}} \label{fig:explode}
	\end{center}
\end{figure}
\subsubsection{PETG}
\begin{figure}[h]
	\begin{center}
		\includegraphics[scale=0.1]{images/DSC_0023.JPG}
		\caption{\small {\it {This is a description}}} \label{fig:explode}
	\end{center}
\end{figure}
\subsubsection{PP-H}
\begin{figure}[h]
	\begin{center}
		\includegraphics[scale=0.1]{images/DSC_0029.JPG}
		\caption{\small {\it {This is a description}}} \label{fig:explode}
	\end{center}
\end{figure}
\subsubsection{POM-C}
\begin{figure}[h]
	\begin{center}
		\includegraphics[scale=0.1]{images/DSC_0039.JPG}
		\caption{\small {\it {This is a description}}} \label{fig:explode}
	\end{center}
\end{figure}
\subsubsection{ACRYL}
\subsubsection{PEEK}
\subsubsection{PPSU}

\subsection{Iterations}
we have used prototyping in order to get a viable device, through the different designs we have been able to see different problems, which have ment that we had to iterate to a new version, we have been limited by time, so we have had to make some compromises

\subsubsection{fisrt}
The first iteration that we build did have some problems.
The first is that it is expensive to build, since we are using a lot of acrylic, the secound point is that it still is heavy, and unwieldy.
But i did give some ideas for the next iteration.

\subsubsection{secound}


\subsubsection{thried}


\subsubsection{fourth}


\subsubsection{fith}

\subsubsection{sixth}

