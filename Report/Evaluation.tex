\subsection{Evaluation}
We did not have the abilility to do testing 


We did not do user test on it, since we did not have acess to medical personal, we did insted stress, visual, sound and power test on the last design, and figured out how much each device will cost in tota and if the budget will hold.

\subsubsection{Break Test}
We tested to see how much the record can handle before it breaks.
de did that by taking the weight in the workshop, that can measure how much kg, something ways,put it on the far end of the acrylic, and pushed down, it measured that it can handle 7 kg of pressure before it breaks, one strange thing was that it did not break where we thought it would break in the sharp corner, we can see that it broke, further out from the center.
this might suggest that the box it self, is helping to give it strength, or that it just was weakest there, the glue might also have some thing to dio with it.
We can do this test over and over again, with different results, since we don't know what factor, had a play in the result.

\subsubsection{Drop Test}
We did testing on the last iteration of our design, where we drop at a distance of 1 meter 10 times to simulate the fall from a desk, after each time we locked for signs that it was about to crack or fracture.
After the 10'th drop we started to see cracks in the acrylic, the cracks appeared in the cutout section where there is a sharp corner, however this was only on the front acrylic plate, had we done the drop many more times we properly would have seen the same result in the back plate.
This is properly due to the laser cutting, and could be avoid by making a round cut rather that a sharp cut, or by drilling a hule to relieve tension there might have been build up, during the laser cutting.

We did not check the hardware during the drop test, afterwards we tested the electronic and found that it hadn't suffered any damage.


\subsubsection{Power Test}
For the power test we borrowed a discharger, from vaidas, where you can set the amount of amps it has to drain.
We calculated that the device would drain around 350 milliamps, if it runs at maximum capacity.
From that we found out that it would run for 4 hours in the worst case, we later found out that if it is in idle mode, it can run for around 20 hours.

\subsubsection{Visual Test}
We tested if we were able to see the different LED colors at a distance of 10 meter or more at night and during the day, throughout the test we had no problems seeing the led change, at distance over 20 meters.

\subsubsection{Sound Test}
In order to see if we where able to find the record only by sound.
We set up some test senarios, 2 where it was out in the open, 2 where it was hidden under some object and 2 where it was inside an object. From the atarting point of the person in the test, there should be the same distance.
At eacy trail we did not find any major difference, whitch conclude that it has a high enof tone to find it.

\subsubsection{Price}
Since we had to keep it under 1000 kr pr device, we locked around on the internet, to finde the cheapest parts.
the price for the acrylic is taken from RIAS \cite{RIAS} where one 3mm thick $1 m^2$ acrylic is 532 kr.
\begin{itemize} \itemsep0em
	 \item WIFI: 191 kr \cite{Adafruit}
	 \item Arduino: 54 kr \cite{Sparkfun}
	 \item RFID: 112 kr \cite{Let-Eletronik}
	 \item Batterie: Courtesy of getvolt  \cite{Getvolt}
	 \item LED: 7 kr \cite{Adafruit}
	 \item Charger: Courtesy of getvolt \cite{Getvolt}
	 \item Stepup converter: Courtesy of getvolt  \cite{Getvolt}
	  \item Acryl: $(acrylic hight * acrylic width)*2 = 0,2 m^2 = 106.4 kr $ \cite{RIAS}
	 \item Total: 470,4 kr
\end{itemize}
If we had bought the stepup, charger and batteri we still would have been under the budgit limit.
