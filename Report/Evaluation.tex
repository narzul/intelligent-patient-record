\subsection{Evaluation}
\subsubsection{Drop Test}
We did testing on the last iteration of our desgin, where we drop at a distance of 1 meter 10 times to simulate the fall from a desk, efter each time we locked for signs that ist was about to crack of fracture.
After the 10'th drop we started to see cracks in the acrylic, the cracks apired in the cutout section where there is a sharp courner, however this was only on the front acrylic plade, had we done the drop many more times we prolery whould have seen the same result in the back plate. 
This is properly due to the laser cutting, and could be avoid by making a round cut rathere that a sharp cut, or by drilling a hule to relive tension there might have been build up, during the laser cuting.

We did not check the hardware during the drop test, afterwards we tested the eletronic and found that it hadn't suffered any damage.

\subsubsection{Power Test}
For the power test we borrowed a discharger, from vaidas, where you can set the amount of amps it has to drain.
We calculated that the devise whould drain around 350 milliamps, if it runs at maximum capacity.
From that we found out that it would rund for 4 hours in the worst case, we later found out that if it is in idle mode, it can run for around 20 hours.

\subsubsection{Price}
Since we had to keep it under 1000 kr pr device, we locked around on the internet, to finde the cheapest parts.
the price for the acrylic is taken from RIAS \cite{RIAS} where one 3mm thick $1 m^2$ acrylic is 532 kr.
\begin{itemize} \itemsep0em
	 \item WIFI: 191 kr \cite{Adafruit}
	 \item Arduino: 54 kr \cite{Sparkfun}
	 \item RFID: 112 kr \cite{Let-Eletronik}
	 \item Batterie: Courtesy of getvolt  \cite{Getvolt}
	 \item LED: 7 kr \cite{Adafruit}
	 \item Charger: Courtesy of getvolt \cite{Getvolt}
	 \item Stepup converter: Courtesy of getvolt  \cite{Getvolt}
	  \item Acryl: $(acrylic hight * acrylic width)*2 = 0,2 m^2 = 106.4 kr $ \cite{RIAS}
	 \item Total: 470,4 kr
\end{itemize}
If we had bought the stepup, charger and batteri we still would have been under the budgit limit.

\subsubsection{LED}
We tested if we ware able to see the different LED colors at a distence of 10 meter or more at night and during the day, through out the test we had no problems seeing the led change, at distance over 20 meters.

\subsubsection{Sound}
We tryed to see if we could hide the device from each other inside pitlab different places, with the sound on.
We where able to find it every time, also when it was hid indeside a closet.
