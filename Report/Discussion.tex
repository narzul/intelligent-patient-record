\section{Results and discussion}

\section{Iterations}

The following five prototype iterations shown on figure \ref{fig:v1} to \ref{fig:v5}, depict an exploded view of the design on the left, and the physical version on the right. There is one major change in design during the process from prototype two to three, but they have all one thing in common: the journal is meant to be fixed on top of an acrylic sheet. \\

The first two prototype ideas shown on figures \ref{fig:v1} and \ref{fig:v2} were made entirely with acrylic sheets cut with a laser-cutter. The first prototype on figure \ref{fig:v1} was designed to accommodate eight banks of 4 x 1,5V AAA batteries. Compartments for each electronic component were made to place them on a flat surface, and reduce thickness. A box on the left of the journal was meant to hold LEDs. This design had many flaws however -- the list below describes the various issues.

\begin{itemize} \itemsep0em
  \item It felt heavy, and the weight exceeded 500g.
  \item The sharp edges from the laser-cut acrylic were slightly uncomfortable.
  \item The sheets were glued together, and made it impossible to open in case of the need for maintenance.
  \item The glueing process is tedious, because of of the use of extra tools such as a clamp, and the hardening takes 24h.
  \item Compartments were engraved with the laser-cutter, which requires constant supervision of the cutting.
  \item Problems with displacement of compartments from the different acrylic sheets.
  \item No charging possibility.
  \item Wiring of components was difficult.
  \item The side piece made it difficult to open the journal.
\end{itemize}

A few positive characteristics were observed:
\begin{itemize} \itemsep0em
  \item Removal of protruding bolts.
  \item Reduced thickness by 15mm.
\end{itemize}


\begin{figure}[h]
\begin{minipage}[b]{7.5cm}
\centering
\includegraphics[scale=0.235]{figures/iterations/v1.png}
\end{minipage}
% \hspace{0.5cm}
\begin{minipage}[b]{7.5cm}
\centering
\includegraphics[scale=0.58]{figures/iterations/v1-photo.jpg}
\end{minipage}
\caption{\small {\it {The first prototype idea}}} \label{fig:v1}
\end{figure}

\clearpage

% -------------------

The second prototype took the previous downsides into consideration, and produced the device shown on figure \ref{fig:v2}. The idea here, is that the electronics were to be moved to a box at the bottom of the journal, in order to make the journal feel less bulky while holding it. \\

Following are the issues found with this design.

\begin{itemize} \itemsep0em
	\item Not very pleasant to hold, as the box holding the electronics ended up getting in the way of the user.
	\item Impossibility of performing maintenance, due to the acrylic being glued.
	\item Structurally unsound design - attachment of box to the sheet was flimsy, and broke easily.
	\item No charging possibility.
	\item The weight distribution would have been very uneven, due to the placement of box at the bottom.
\end{itemize}

\begin{figure}[h]
\begin{minipage}[b]{7.5cm}
\centering
\includegraphics[scale=0.235]{figures/iterations/v3.png}
\end{minipage}
% \hspace{0.5cm}
\begin{minipage}[b]{7.5cm}
\centering
\includegraphics[scale=0.58]{figures/iterations/v3-photo.jpg}
\end{minipage}
\caption{\small {\it {The second prototype idea}}} \label{fig:v2}
\end{figure}

This prototype was discarded before an attempt to place the electronics in the box, due to the obvious disadvantages. The technique of laser-cutting acrylic sheets and using glue to attach them together proved to be a slow and tedious task. Due to the difficulty of using screws to assemble the parts and using glue instead, made the designs unmaintainable. Complex laser-cut acrylic parts also tended to become brittle, so a rethinking of the design was in order. Using acrylic wasn't out of the question, but the complexity of those parts had to be reduced, and the use of 3D printed PLA for those parts were introduced in the following prototype.

% -------------------

\begin{figure}[h]
\begin{minipage}[b]{7.5cm}
\centering
\includegraphics[scale=0.235]{figures/iterations/v4.png}
\end{minipage}
% \hspace{0.5cm}
\begin{minipage}[b]{7.5cm}
\centering
\includegraphics[scale=0.58]{figures/iterations/v4-photo.jpg}
\end{minipage}
\caption{\small {\it {The third prototype idea}}} \label{fig:v3}
\end{figure}

% -------------------

\begin{figure}[h]
\begin{minipage}[b]{7.5cm}
\centering
\includegraphics[scale=0.235]{figures/iterations/v5.png}
\end{minipage}
% \hspace{0.5cm}
\begin{minipage}[b]{7.5cm}
\centering
\includegraphics[scale=0.58]{figures/iterations/v5-photo.jpg}
\end{minipage}
\caption{\small {\it {The fourth prototype idea}}} \label{fig:v4}
\end{figure}

% -------------------

\begin{figure}[h]
\begin{minipage}[b]{7.5cm}
\centering
\includegraphics[scale=0.235]{figures/iterations/v6.png}
\end{minipage}
% \hspace{0.5cm}
\begin{minipage}[b]{7.5cm}
\centering
\includegraphics[scale=0.58]{figures/iterations/v6-photo.jpg}
\end{minipage}
\caption{\small {\it {The fifth prototype idea}}} \label{fig:v5}
\end{figure}

% -------------------

\clearpage

\subsection{Evaluation}

The first three iterations weren't evaluated with the methods described in section \ref{section-evaluation}, as the prototypes weren't deemed to be functional. The evaluation therefore didn't involve any users, because technical issues had to be fixed. However, it can still be argued that the pattern of prototyping, reviewing, and refining took place, as illustrated by the steps in figure \ref{fig:eng-design-process}. Only the fourth prototype was deemed to be worthy of usability-testing with real users. \\

Five separate users were given the prototype shown on figure \ref{fig:journal-1} to hold in their hands, and were asked some questions, in order to perform the thinking-aloud protocol described in section \ref{section-evaluation}. The following is the response gathered from that round of evaluation:

\begin{itemize}
	\item Test person A:
		\begin{itemize} \itemsep0em
	  		\item Q: What do you think of the design?
	  		\item A: It is not so nice, the box sticks out too much.
	  		\item Q: What do you think about the weight?
			\item A: The weight seems fine to me.
			\item Q: Is it easy to hold?
			\item A: It's easy to hold but, but it would be best if the box would be the same height as the plate.
			\item Q: Any comments?
			\item A: 
		\end{itemize}

	\item Test person B:
		\begin{itemize} \itemsep0em
			\item Q: What do you think of the design?
			\item A: It is very fine, even if there is color difference between the box and the plate.
			\item Q: What do you think about the weight?
			\item A: It is ok.
			\item Q: Is it easy to hold?
			\item A: Yes you can hold it, but it will probably be a bit of a problem if you are left-handed.
			\item Q: Any comments?
			\item A: It would be nice if the box could be slimmer and have the same height as the plate.
		\end{itemize}

		\vspace{5mm}

	\item Test person C:
		\begin{itemize} \itemsep0em
			\item Q: What do you think of the design?
			\item A: The plates are a little thick, and it would be best if you could remove one.
			\item Q: What do you think about the weight?
			\item A: It is too heavy, if you remove one of the plates you could make it lighter.
			\item Q: Is it easy to hold?
			\item A: The box gets a bit in the way, the edges are a bit sharp too.
			\item Q: Any comments?
			\item A: 
		\end{itemize}

	\item Test person D:
		\begin{itemize} \itemsep0em
			\item Q: What do you think of the design?
			\item A: I think it's very neat.
			\item Q: What do you think about the weight?
			\item A: I think it's fine.
			\item Q: Is it easy to hold?
			\item A: Yes.
			\item Q: Any comments?
			\item A: Could be slimmer.
		\end{itemize}

	\item Test person E:
		\begin{itemize} \itemsep0em
			\item Q: What do you think of the design?
			\item A: It's great, I like the rounded corners.
			\item Q: What do you think about the weight?
			\item A: It fits me.
			\item Q: Is it easy to hold?
			\item A: Yes and no, if I hold it at the left side then yes.
			\item Q: Any comments?
			\item A: It must be made so you can hold it equally comfortably with both hands.
		\end{itemize}					
\end{itemize}

\clearpage

Some recurring comments can be gathered from this think-aloud test, and the improvement suggestions can be summarised:
\begin{itemize} \itemsep0em
	\item The electronics compartment should be thinner.
	\item The electronics compartment should be placed out of the way of where the user holds the journal.
\end{itemize}

Although the test revealed the prototype to be quite pleasing, people have very high expectations from electronics, and expect an electronic device to be very thin, as this seems to be the trend with computers and smartphones. \\

The users didn't come with many free comments however -- they should have been asked more questions, such as the readability of the LEDs. Unfortunately, the development had to stop for this project due to time constraints, and a fifth iteration wouldn't have been possible to make. If more time had been at disposal, the thinking-aloud test would have been the conclusion of a first cycle of development. The results from the test would have been a kickstarter for a new brainstorming session about how to solve those issues, and to build a new prototype to be evaluated. \\

The point is to run multiple tests because after the first study with 5 users has found 85\% of the usability problems, a second test will discover whether the fixes worked or whether they didn't. And when introducing a new design, there is the risk of introducing a new usability problem, even if the old one did get fixed. The second test will therefore also serve as quality assurance of the outcome of the first study. The same insight applies to this redesign: not all the fixes will work, and some deeper issues will be uncovered after cleaning up the previous prototype, and a third design cycle is needed. \\

Another usability test that could have been performed, is the use of scenarios. A task scenario is an action that is asked to the participant to take, on the prototype under test. Task scenarios need to provide context, so users engage with the device, and pretend in the case of this project, that they are a doctor or a nurse. Watching people use the prototype, would have been an effective way of understanding the positive and negative aspects of it.