\clearpage

\section{Materials and methods}
\subsection{Materials}

During the creation of a prototype, the tools and materials at disposition were the following:

\begin{itemize}
	\item Software:
		\begin{itemize} \itemsep0em
	  		\item OpenSCAD: software for creating solid 3D CAD objects
  			\item Slic3r: G-code generator for 3D printers
  			\item Inkscape: vector graphics editor
		\end{itemize}

	\item Hardware:
		\begin{itemize} \itemsep0em
	  		\item MAXbot 3D printer
	  		\item Epilog Helix 24 laser cutter
	  		\item PLA filament for 3D printer
	  		\item Acrylic sheets for laser cutter
		\end{itemize}
\end{itemize}


\subsection{Methods}

This section will describe the application of the steps 2 to 5, shown on figure \ref{fig:eng-design-process}. After the probem had been defined, a brainstorming session was made in order to generate ideas and select a possible solution to the problem. Two main points were discussed, that is, what should the device look like, and what techniques are going to be employed in the building process. \\

When taking the requirements into consideration, the first step taken was to draw a rough sketch on paper, before creating a virtual model in a CAD application. The representation of the first prototype was therefore in the form of a pen and paper sketch, which is defined as offline, low-fidelity, fixed i.e. non-interactive, and rapid, as described in section \ref{section-prototyping}. The sketch saves time and money since it enables creating a good understanding of the product before development of a more high-fidelity prototype. Once the design was agreed upon through the help of the sketch, a physical model was built. Here, the prototype could be described as being high-fidelity, open, and rapid. This type of protyping was used for the four iterations decribed in the next section. \\

For the creation of a high-fidelity physical part, a blend of laser-cutting and 3D printing was used. OpenSCAD was used to draw a 3D model of the prototype. It consisted of two different parts, one for the cutting of acrylic sheets, and the other for 3D printing in PLA. The model destined for laser cutting was exported from OpenSCAD to .DXF file, and imported into Inkscape, in order to export it again in PDF format, as the laser cutter uses that format as input. \\

The last prototype was evaluated using the thinking-aloud method described in section \ref{section-evaluation}. Five separate users were given the prototype shown on figure \ref{fig:journal-1} to hold in their hands, and were asked some questions. The observer asking the questions recorded the user-feedback, and transcribed the results on paper. 














% \subsection{Software}

% For the Design we have used openscad, and steard away from 3d design programs like 3d studio max and blender.
% The reason is when you design a file in 3d studio max or blender, it might look big, and you may have followede the messurments in the program, you usual have to multiply the figur by a factor of 1000x due to conversion problems. Beside that you can have problems with the node points in the design, some times the design file might lock like all the points that need to be conected are, but when you multiply the figure by a factor or zoom up close, that might not be true, and that can give problems.
% The othere is the avalibity and price, openscad is a simple, lightwitght and opensource program that is easy to learn and use for programmers, since you create objects if by writing simple code ex. Cube([5,5,5]); 
% We could have bought a linces for some other program like autocad or different cad program, but we found the unessery since it is simple 3D models, and we don't need a huge program for that.

% When the design is done we have exportede the openscad file to STL, that we have but into scli3r, that are one of the best stl to gcode convert there exsist, where you take stl files that are exportede from openscad and convert them to gcode, which is instruction code that the firmware on the 3D printer understand. beside that it is openscource, and is used for almost all 3d printeres.

% For the laser cutting we have used inkscabe, which also is an opensource program, the reason is that we only had to change the line thickness on the DXF file that was exported from openscad. and save it as an pdf. 

% \subsection{Machines}
% \subsubsection{3D Printers}
% There is a lot of different 3D printers design, the printers that we had access to was prusa i3 at ITU, and MAXbot in labitat, they work in the same way, with it's x,y  and z axsis, the main differense is the thickness of the pla that they use, geearing and how the bed is cunstructed. The prusa i3 at ITU usses 3mm pla vs MAXbot that uses 1.75 mm pla, this has the complecation that prusa need to have a gearing system in order to give enough pressur.
% The last main difference is that the Prusa i3 don't have any springs in the bed, which can leed to some problems where the nossel get attach ot the print and force it of the plate.

% \subsubsection{Laser cutter}
% For the laser cutting we have used the 60W laser at ITU, we did have some back up at some different fablabs, if the laser stop working.

% \section{Design specifications}
% Explanations of how to build the product, including information such as:
% - System architecture
% - Drawings and sketches
% - Parts and supply ordering information

% \subsection{Hardware}
% Arduino
% Wifi
% RFID
% Buzzer
% Led
% Capicitor
% Battery
% Battery charger