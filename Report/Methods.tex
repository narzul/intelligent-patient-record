\section{Methods}
\subsection{Software}
\subsubsection{Openscad}
\subsubsection{Inkscab}
\subsubsection{Slic3r}
For the Design we have used openscad, and steard away from 3d design programs like 3d studio max and blender.
The reason for that is when you design a file in 3d studio max or blender, it might look big, and you may have followede the messurments in the program, you usual have to multiply the figur by a factor of 1000x due to conversion problems. Beside that you can have problems with the node points in the design, some times the design file might lock like all the points that need to be conected are, but when you multiply the figure by a factor or zoom up close, that might not be true, and that can give problems.
The othere if the avalibity and price, openscad is simple, lightwitght and opensource program that is easy to learn and use for programmors, since the way you create objects if by writing simple code ex. Cube([5,5,5]); now compaired agienst autocad and othere cad programs we don't have the money to by a license.
beside openscad we have also used inkscabe, the reason that we have used this insted of illustrator is that it is free, and it does exactly what we need, for the laser cutting.
we have also used scli3r since it is the best stl to gcode convert there exsist, and it is openscource, and is used for almost all 3d printeres.

\subsection{Machines}
in the undertaking of this project we have used different tools
3D Printers
Laser cutter
Soldering tool

- materials description
- 